\documentclass[a4paper,12pt]{article}
\usepackage[utf8]{inputenc}
\usepackage{amsmath}
\usepackage[table,xcdraw]{xcolor}
\usepackage{graphicx}
\usepackage{amsmath}
\usepackage[bottom=2.0cm,top=2.0cm,left=2.0cm,right=2.0cm]{geometry}
\usepackage[brazilian]{babel}
\usepackage{indentfirst}
\usepackage{hyperref}  
\usepackage[nottoc]{tocbibind}
\usepackage{lipsum}
\usepackage{blindtext}
\usepackage{fontspec}
\usepackage[acronym, toc]{glossaries}
\usepackage{array}
\usepackage{multirow}
\usepackage{pdfpages}
\usepackage{tabularray}
\usepackage{siunitx}
\usepackage{float}
\usepackage{lscape}
\usepackage[table,xcdraw]{xcolor}

\hypersetup{colorlinks,citecolor=black,filecolor=black,linkcolor=black,urlcolor=black}
\setmainfont{Arial}
\linespread{1.25}
\parindent=1.25cm
\makeglossaries

\renewcommand{\labelenumii}{\theenumii}
\renewcommand{\theenumii}{\theenumi\arabic{enumii}.}

\begin{document}

\title{Requisitos para desenvolvimento}

\begin{titlepage}
	
		\begin{figure}[H]
		    \begin{center}
		        \includegraphics[width=7cm]{logouf.png}
		    \end{center}
		\end{figure}
	
	\begin{center}
        \vspace{-1cm}
        \large{\textbf{Universidade Federal da Paraíba - UFPB}}\\
        \large{Centro de Energias Alternativas e Renováveis - CEAR}\\
        \large{Departamento de Engenharia Elétrica - DEE}\\
        \large{Disciplina de Informática Industrial}\\
        
        \vspace*{\fill}
        \Large\textbf{Requisitos para desenvolvimento}
        \vspace*{\fill}
        
        \normalsize
        \hfill Grupo: \\
        \hfill Jaedson Barbosa Serafim \hspace{20pt} Mat: 20170024577\\\hspace{20pt} 
        \hfill Jose Helio Bento da Silva \hspace{20pt} Mat: 20180048133\\\hspace{20pt} 
        \hfill Fabiano de Souza Chaves Colaço \hspace{20pt} Mat: 20200093642  \newline
        
        \hfill Professor orientador:\\
        \hfill Ademar Virgolino da Silva Netto
        
        
        \vspace{\fill}
        João Pessoa - PB\\
        \today
          
	\end{center}
\end{titlepage}

\newpage

%\listoffigures
%\clearpage
\listoftables
\clearpage
\tableofcontents
\newpage
\pagenumbering{arabic}
\pagebreak

\section{Introdução}

A parceria da Huawei com a Universidade Federal Paraíba (UFPB) resultou na construção de uma usina fotovoltaica voltada para fins educacionais e econômicos no Centro de Informática, Campus I. A usina foi projetada tomando como modelo o parque solar de Coremas, possuindo 372 painéis conectados em doze strings, com capacidade máxima de 240kWp.

Com o intuito de maximizar a potência gerada, o projeto incluiu a capacidade das placas de ajustarem a sua inclinação para aumentar a incidência solar, a planta possui quatro trackers, cada um conectado à três strings.

São utilizados dois métodos para controlar a posição dos trackers, um comercial e outro desenvolvido por estudantes da UFPB. A solução comercial define a angulação dos painéis a partir do horário, já a solução desenvolvida pelos estudantes utiliza uma inteligência artificial para ajustar a angulação dos painéis para a que maximize a potência gerada. Metade dos trackers utilizam a solução comercial e a outra a solução desenvolvida pelos estudantes.

Na planta são utilizados dois inversores SUN2000-100KTL-M1 de 100kW.  A ligação das strings nos inversores foi separada pela forma de controle da angulação das strings, em um inversor são ligadas as strings controladas pelo método comercial e no outro inversor são conectadas as strings controladas pelo método desenvolvido pelos estudantes.

Além da parte relacionada com a geração da energia, existe uma estação meteorológica, entre os dados obtidos por ela temos: velocidade e direção do vento, temperatura, radiação total e difusa, humidade e precipitação da chuva. Esses dados são utilizados para o controle dos trackers, além disso, eles são armazenados em um banco de dados para futuras análises.

O controle da usina é feito através do Network Control Unit (NCU), a comunicação entre ele e os demais componentes da usina (inversores, trackers e estação meteorológica) é feita através do protocolo de comunicação Modbus TCP IP.

%\section{Glossário}

\section{Definição dos Requisitos de Usuário}
\begin{enumerate}
    \item O sistema deve acessar coletar os dados da estação meteorológica listados na tabela 1, os dados da usina listados na tabela 2 e oss dados do encoder do tracker. 
    \item O sistema deve quantificar a potência e energia gerada por string, a máxima diária, a média diária e a mínima diária
    \item O sistema deve quantificar a potência e energia máxima mensal, a média mensal e a mínima mensal
    \item Deve quantificar a potência e energia máxima anual, a média anual e a mínima anual
    \item O sistema deve separar os três itens anterior da técnica de controle privada e da técnica de  controle da UFPB
    \item Deve ser calculado o rendimento diário, mensal e anual em reais da geração tanto para técnica de controle privada quanto da UFPB
    \item Deve mostrar o status da usina: gerando, em manutenção
    \item Deve mostrar se tem alguma técnica de backtraking
    \item Os dados meteorológicos devem ser mostrados em um período de x minutos
    \item O sistema deve conter uma tela geral na qual mostra os principais dados e gráficos de tendências
    \item Deve conter uma tela para ilustrar os demais gráficos
    \item Deve ter uma tela para os dados dados meteorológicos
    \item Deve conter uma tela para os dados históricos 
    \item Deve ter uma tela para geração de relatórios
    \item Deve ter uma tela de login

    \item O sistema deve conter um histórico de eventos  para armazenar ações ocorridas durante o processo e listadas na tabela 3
    
    \item Conter uma base de dados para armazenar usuário e senha
    \item Conter um formulário de cadastro
    \item conter um sistema de confirmação de email e/ou número
    \item Conter um formulário de login
    \item conter um sistema de verificação de email e senha
    \item Conter uma página de destino
    \item Conter uma página de configuração para mudança de email e senha
    
    \item O sistema deve ter um sistema de Alarme cujas causas estão listadas na tabela 4
    
    \item O sistema deve ter um sistema de alerta
    \item O sistema deve analisar os dados de geração e avaliar a necessidade de manutenção
    \item O sistema deve ser capaz de controlar o ângulo do tracker
    \item O sistema deve gerar relatórios de eventos
    \item O sistema deve ter um banco de dados para armazenamento de dados pré-determinados

\end{enumerate}


    \begin{table}[htbp]
    \begin{center}
    \begin{tabular}{|c|}
    \hline
       \textbf{Estação meterológica} \\ \hline
        Radiação difusa (W/$m^2$)\\
        Radiação total (W/$m^2$) \\
        Direção do vento \\
        Velocidade do vento (m/s)\\
        Temperatura ambiente (°C)\\
        Precipitação da chuva (mm/$m^2$)\\
        Temperatura de uma placa (°C) \\
        \hline
    \end{tabular}
    \caption{Dados fornecidos pela estação meterológica }
    \end{center}
    \end{table}

    \begin{table}[htbp]
    \begin{center}
    \begin{tabular}{|c|}
   
    \hline
       \textbf{Usina } \\ \hline
        Potência (W)\\
        Tensão (V) \\
        Corrente (A) \\
    
        \hline
    \end{tabular}
    \caption{Dados fornecidos pelos inversores de frequência }
    \end{center}
    \end{table}
  
     \begin{table}[htbp]
    \begin{center}
    \begin{tabular}{|c|}
   
    \hline
       \textbf{Ocorrências } \\ \hline
        Hora de ocorrência de um alerta\\
        Hora em que o alerta foi resolvido \\
        Hora de impressão de relatório \\
        Hora de login e logout\\
    
        \hline
    \end{tabular}
    \caption{Ocorrências a serem gravadas }
    \end{center}
    \end{table}   



    \begin{table}[htbp]
    \begin{center}
    \begin{tabular}{|c|}
   
    \hline
       \textbf{Histórico } \\ \hline
        ângulo do tracker fora do limite de segurança\\
        os inversores não estiver funcionando corretamente \\
        ângulo do tracker estiver muito distante do estabelecido \\
        sensores não estiverem funcionando corretamente \\
        Se houver sobre ou subtensão e sobre ou subcorrente no tracker \\
    
        \hline
    \end{tabular}
    \caption{Causas de alarmes }
    \end{center}
    \end{table}   

\end{document}
